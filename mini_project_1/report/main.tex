\section*{Project Goals}
The goal of this project is to create a USB powered circuit to flash an LED with a period of 1s $\pm$ 10\%.
This translates to a frequency of 1Hz $\pm$ 10\%
The components given makes this doable by creating a hysteretic oscillator using an op-amp.

\section*{Design Decisions}

\subsection*{Oscillator Component Value Selection}

Designing this oscillation circuit required picking values for $V_{ref}$ the various feedback resistors and capacitors.\\

Having $V_{ref}$ set at $\frac{1}{2}V_{dd}$ allows for the upper and lower oscillator
trigger thresholds to be centered in the $V_{dd}$ range when the voltage divider
resisters, $R`$' connected to the inverting op-amp input are selected to be equal.
In this arrangement, $RC=\tau$, connected to the non-inverting op-amp input, are most important
to controlling the timing of the circuit. I selected to use the $0.1\mu F$ capacitor for $C$ because
the the op-amp MCP6022 IC uses a $0.1\mu F$ capacitor for its power bypass capacitance meaning
the design already needs a capacitor of that size which minimized the variety of capacitor values needed.
Additionally, the $0.1\mu F$ capacitor has a 5\% tolerance which is less than the $1\mu F$ capacitor which
has a 10\% tolerance. This makes for less variation in the output frequency due tolerances.
With $C$ fixed, I needed to solve for $R$ and did so using the equation below in which
$V_{up}$ is the threshold in which the op-amp output goes high, $V_{dn}$ is the
threshold in which the op-amp output goes low, $T_l$ is the period in which the oscillator outputs
low.


\[
V_{up} = V_{dn}e^{\frac{-T_l}{\tau}}
\]\\

With $V_{ref} = \frac{1}{2}V_{dd}$, $V_{up} = \frac{1}{4}V_{dd}$ and $V_{dn} =  \frac{3}{4}V_{dd}$. 
Solving for $\tau$ we get
\[
\tau = \frac{-T_l}{log \left( \frac{V_{up}}{V_{dn}} \right)}
\]

Simplifying further:
\[
\tau = \frac{-T_l}{log \left( \frac{1}{3} \right)}
\]\\

Breaking $\tau$ into $RC$ and dividing out $C$ solves for $R$:

\[
R = \frac{-T_l}{log\left( \frac{1}{3} \right)C}
\]\\


I want the design to have a 50\% duty cycle so the low time should be equal
to the high time. With the design needing to have a 1Hz frequency, the high
time and low time each should be 0.5s. As such $T_l = 0.5s$ and knowing that
$C$ is selected to be $0.1\mu F$, the nominal value for $R = 4.55M \Omega$. A 
$4.55M\Omega$ resister is not available in the BOM but it can be closely created
using a series connection of two $2M\Omega$, one $499k\Omega$, and one $49.9k\Omega$ resistors.
This creates an equivalent resistance of $4.5489M\Omega$ which is less than 0.03\% different 
than the nominal $4.55M\Omega$.\\

The circuit runs on a single ended 3.3V supply derived from the $V_{bus}$ of a USB using a linear regulator.
As such $V_{dd}$ for this circuit is 3.3V and since I want to have a $V_{ref}$ equal to $\frac{1}{2}V_{dd}$,
I need to divide the $V_{dd}$ supply in two. I did this using a voltage divider and two $1k\Omega$ resistors.
I did not directly attach the $V_{out}$ of the voltage divider to the feedback on the op-amp though. 
Doing this would have put the voltage divider as part of the feedback causing the 1.65V reference
to vary. To mitigate this I used the second op-amp on the MCP6022 to create a buffer. I then connected
the output of that op-amp for $V_{ref}$. 

\subsection*{IC Application}
The datasheet for the MCP1702 linear regular suggests having $1\mu F$ bypass capacitors on the 
input and output so these were included in the design. Additionally the MCP6022 op-amp datasheet suggests
having a $0.1\mu F$ bypass capacitor within 2mm and $1\mu F$ bulk capacitor within 100mm.
As such $0.1\mu F$ cap is connected less than 2mm away from the power pin and the $1\mu F$ bypasss
capacitor on the output of the linear regular is within 100mm and acts as bulk capacitance
for the op-amp

\subsection*{Worse Case Analysis}

I used LTSpice to do a worst-case analysis on the circuit. The simulation includes tolerance for
each component and simulates the circuit against the tolerances of one another.\\

The LTSpice schematic can be seen below:

 \begin{figure}[htbp]
  \centering
  \includegraphics[width=0.8\textwidth]{assets/worst_case_schematic.png}
  \caption{}
  \label{fig:worst_case_sim_schematic}
\end{figure}


\newpage
I exported the raw data for processing using Python. I created a script that uses the rising edges
to calculate frequency of the signals, determines the highest and lowest frequencies of all the
runs/combination, and plots all the runs on a histogram to show a distribution. My simulation uses 11 components so there
were 2028 combination. Out of these, the highest frequency was 1.091Hz and the lowest was 0.935Hz
both of which are within the 1Hz $\pm$ 1\% requirement.\\

The distribution of frequencies can be seen below:


\begin{figure}[htbp]
  \centering
  \includegraphics[width=0.8\textwidth]{assets/worst_case_frequency_distribution.png}
  \caption{}
  \label{fig:worst_case_frequency_distribution}
\end{figure}

\newpage


\subsubsection*{Model Abstractions}
One model abstraction in this simulation is not including the linear regulator. We were told that
we do not need to model the tolerance of the linear regulator so I chose to generate the 3.3V power
using a standard LTSpice voltage source. Since I didn't include the linear regulator directly, it also 
means that the bypass capacitor on the input of the linear regulator is not included.\\

The simulation also does not include tolerance for the current limiting resistor for the LED as suggested.

\subsection*{Layout Considerations}
I followed the convention of having 8mil trace widths for signals and 20mil power trace widths were possible.
I was able to follow this except for trace connecting to the positive rail on the op-amp. I found that 20mils was
too wide and I needed to reduce it to 17mil. Although reduced in size, the current flow through that trace is still
well below the max current flow for a 17mil trace so this should not cause any problems. The smaller trace width
likely has a reduction in thermal capacity and bulk capacitance but not enough to cause concern for a simple circuit
like this. I also used the suggested 24mil via with 12mil drill hole diameter.\\


I tried to focus on minimizing trace lengths and keeping routed tracing far from one another to minimize crosstalk.
The frequencies in this circuit are low enough that such considerations likely are overkill but I still
wanted to try to follow best practices. I tried to keep power traces short and signals traces also short and away
from unlike signals. I also tried to place vias in position to minimize loop area.
My layout likely could have been made more compact but it would have been at the trade
off of routing. Additionally, I wanted to consider manufacturing and assembly. Having components/traces very close to
each other can make it more difficult for fabs to fabricate for a low cost and having sufficient space between can make its
more viable to fabricate these on a smaller desktop style PCB mills. Additionally having done hand soldering with 
SMD I've learned that having little space between components increases the likelihood of solder on one
pad bridging to another so the extra spacing should be helpful. I also wanted to keep sufficient space for component labels.
